%%%%%%%%%%%%%%%%%%%%%%%%%%%%%%%%%%%%%%%%%%%%%%%%%%%%%%%%%%%%%%%%%%%%%%%%%%%%%%%%
%2345678901234567890123456789012345678901234567890123456789012345678901234567890
%        1         2         3         4         5         6         7         8

\documentclass[letterpaper, 10 pt, conference]{ieeeconf}  % Comment this line out
                                                          % if you need a4paper
%\documentclass[a4paper, 10pt, conference]{ieeeconf}      % Use this line for a4
                                                          % paper

\IEEEoverridecommandlockouts                              % This command is only
                                                          % needed if you want to
                                                          % use the \thanks command
\overrideIEEEmargins
% See the \addtolength command later in the file to balance the column lengths
% on the last page of the document



% The following packages can be found on http:\\www.ctan.org
%\usepackage{graphics} % for pdf, bitmapped graphics files
%\usepackage{epsfig} % for postscript graphics files
%\usepackage{mathptmx} % assumes new font selection scheme installed
%\usepackage{times} % assumes new font selection scheme installed
%\usepackage{amsmath} % assumes amsmath package installed
%\usepackage{amssymb}  % assumes amsmath package installed

\title{\LARGE \bf
Detection of design pattern : A systematic mapping study}

%\author{ \parbox{3 in}{\centering Huibert Kwakernaak*
%         \thanks{*Use the $\backslash$thanks command to put information here}\\
%         Faculty of Electrical Engineering, Mathematics and Computer Science\\
%         University of Twente\\
%         7500 AE Enschede, The Netherlands\\
%         {\tt\small h.kwakernaak@autsubmit.com}}
%         \hspace*{ 0.5 in}
%         \parbox{3 in}{ \centering Pradeep Misra**
%         \thanks{**The footnote marks may be inserted manually}\\
%        Department of Electrical Engineering \\
%         Wright State University\\
%         Dayton, OH 45435, USA\\
%         {\tt\small pmisra@cs.wright.edu}}
%}

\author{Pierre Gerard, Alexandre Saint-Louis Fortier, Badr Mai and Houari Sahraoui \\
Département IRO \\
Université de Montréal, Montreal, Canada \\
\{gerardpi,Unknown,Unknown,sahraouh\}@iro.umontreal.ca
% <-this % stops a space
%\thanks{*This work was not supported by any organization}% <-this % stops a space
%\thanks{$^{1}$H. Kwakernaak is with Faculty of Electrical Engineering, Mathematics and Computer Science,
 %       University of Twente, 7500 AE Enschede, The Netherlands
  %      {\tt\small h.kwakernaak at papercept.net}}%
%\thanks{$^{2}$P. Misra is with the Department of Electrical Engineering, Wright State University,
   %     Dayton, OH 45435, USA
    %    {\tt\small p.misra at ieee.org}}%
}


\begin{document}



\maketitle
\thispagestyle{empty}
\pagestyle{empty}


%%%%%%%%%%%%%%%%%%%%%%%%%%%%%%%%%%%%%%%%%%%%%%%%%%%%%%%%%%%%%%%%%%%%%%%%%%%%%%%%
\begin{abstract}

Abstract goes here

\end{abstract}


%%%%%%%%%%%%%%%%%%%%%%%%%%%%%%%%%%%%%%%%%%%%%%%%%%%%%%%%%%%%%%%%%%%%%%%%%%%%%%%%
\section{INTRODUCTION}

Introduction goes here
% pourquoi la detection

% pourquoi ce systematic mapping

\section{SYSTEMATIC MAPPING PROCESS}

In this section, we discuss step by step how we conducted our systematic mapping study. We have followed the process defined by Petersen et al \cite{c1}. 

\subsection{Research question}

The main goal of this paper is to determine the quantity and trends of research in design pattern detection and the quality of proposed detection methods.
\begin{itemize}
	\item \textbf{RQ1} : How mature is research on the subject of design pattern detection ?
	\item \textbf{RQ2} : What methods of design pattern detection are used ?
\end{itemize}


\subsection{Data source and queries}

We decided to query only one database, Scopus \cite{c2}, that claimed to be the largest abstract and citation database of peer-reviewed literature. It indeed index the main Computer Science databases : Springer, IEEE, ACM and others. The main advantage of doing so is that one could easily export result of an important number of publication from Scopus.

From the two research question we tried to maximise the number of publication found on the subject. To achieve that purpose we used for the query a disjonction of thesaurus for detection on their respective contracted form : detect*, recogni* and ident*. We have then added to the query a disjonction of pattern and motif.
We have limited the query to the abstract, the title, and keywords of each article.

We have limited our research to the year between 2000 and 2015 included. 

We have also limited our research to the field of computer science and engineering.

The query resulted of a total of 403 articles. In those articles, we found two duplicates that we have immediately removed from the result.

\subsection{Screening}

After having queried the database, we exported results found and process to screening. 

% par paire de deux et resolve conflict

The aim of screening is to select papers relevant to design pattern detection. For that purpose, we used an exclusion scheme containing four criteria :

\begin{enumerate}
	\item \textbf{Not a contribution in software engineering} : %explaination ,
	\item \textbf{Not a full conference or research paper},
	\item \textbf{Not about software design pattern} : %explaination,
	\item \textbf{The main contribution of the paper is not about detection of design pattern} : %explaination
\end{enumerate}

If in article matched more than one criteria, we assigned the lowest criteria to him.

% kappa de cohen

\subsection{Classification scheme}

% comment on l'a choisit

% explication pour chaque catégorie
\begin{itemize}
	\item \textbf{Detection strategy}
	\item \textbf{Language generality}
	\item \textbf{Analysis type}
	\item \textbf{Validation method}
	\item \textbf{Detection Level}
	\item \textbf{Detected pattern types}
	\item \textbf{Pattern detection generality}
\end{itemize}


\subsection{Systematic map}

\section{SELECTION PROCESS RESULT}


\section{CONCLUSIONS}

A conclusion section 



\addtolength{\textheight}{-12cm}   

% This command serves to balance the column lengths
                                  % on the last page of the document manually. It shortens
                                  % the textheight of the last page by a suitable amount.
                                  % This command does not take effect until the next page
                                  % so it should come on the page before the last. Make
                                  % sure that you do not shorten the textheight too much.

%%%%%%%%%%%%%%%%%%%%%%%%%%%%%%%%%%%%%%%%%%%%%%%%%%%%%%%%%%%%%%%%%%%%%%%%%%%%%%%%



%%%%%%%%%%%%%%%%%%%%%%%%%%%%%%%%%%%%%%%%%%%%%%%%%%%%%%%%%%%%%%%%%%%%%%%%%%%%%%%%



%%%%%%%%%%%%%%%%%%%%%%%%%%%%%%%%%%%%%%%%%%%%%%%%%%%%%%%%%%%%%%%%%%%%%%%%%%%%%%%%
\section*{APPENDIX}

Appendixes should appear before the acknowledgment.

\section*{ACKNOWLEDGMENT}

Acknowledgement


\begin{thebibliography}{99}

\bibitem{c1} K. Petersen, R. Feldt, S. Mujtaba, and M. Mattsson, “Systematic mapping studies in software engineering,” in Proc. of the 12th Int. Conf. on Eval. and Asses. in Soft. Eng., ser. EASE’08. Swinton, UK, UK: British Computer Society, 2008.

\bibitem{c2} www.scopus.com







\end{thebibliography}




\end{document}
